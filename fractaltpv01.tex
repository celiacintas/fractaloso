\documentclass[12tp]{article}
\textwidth = 16cm
\usepackage[utf8]{inputenc}
\usepackage[spanish]{babel}
\usepackage{fancyhdr}
\usepackage{setspace}
\usepackage{minted}
\usepackage[pdftex]{hyperref}
\usepackage[colorinlistoftodos, textwidth=4cm, shadow]{todonotes}
\usepackage{graphicx}  % Inclusión de imágenes
\usepackage{anysize}

\marginsize{3cm}{2cm}{2cm}{2cm}
\renewcommand{\thefootnote}{\Roman{footnote}}
\hypersetup{colorlinks=false,linkcolor=black,citecolor=black,hidelinks}

%%%%%scabecera
%\renewcommand{\headrulewidth}{0.4pt} % grosor de la línea de la cabecera
%\renewcommand{\footrulewidth}{0.4pt} % grosor de la línea del p
\renewcommand{\baselinestretch}{0.97}

\begin{document}
\pagestyle{empty} % seleccionamos un estilo

  \title{\bf {Generación de Galaxias Espirales mediante Automatas Celulares.}\\
  \small{Modelos Fractales y Sistemas Caóticos en Física Computacional y Ciencias Naturales.}} 
%Universidad Nacional de la Patagonia San Juan Bosco \\[0.1cm] Blvd. Brown 3150 - Puerto Madryn (9120) – Chubut \\ 
%Tel-Fax: (2804) 472885 / 450272}  }
  %\subtitle{ Grupo de Investigación de Procesamiento de Imágenes, Facultad de Ingeniería, Universidad Nacional de la Patagonia San Juan Bosco}
  %\author{Cintas Celia, Delrieux Claudio, Bianchi Gloria, Defossé Nahuel }
  \author{
    \small{Cintas Celia}\\
    \small{cintas.celia@gmail.com}
  \and
    \small{Delrieux Claudio}\\
    \small{cad@uns.edu.ar}
    %\and
    %\small{Gonzalez-José Rolando}\\
    %\small{rolando@cenpat.edu.ar}
   % \and
    %\small{Defossé Nahuel}\\
    %\small{snahuel.defosse@gmail.com}
}

%\date{}
%\thispagestyle{empty}
\maketitle

\begin{abstract}
%En esta monografía se trata de trabajar conceptos de espacio-tiempo fractal y relativo a la escala, ambos tópicos trabajados por L. Nottale 
%\footnote{Poner biografia de Nottale} y como se relacionan con el actual entendimiento del tiempo-espacio "Einsteniano".\\
%Por lo que este texto pasará superficialmente por temas que conciernen a la Teoría de la Relatividad General, luego trabajaremos sobre noción de relativo a la escala
%acercandonos a la definición de fractal y su aporte en estas teorias emergentes.


\textbf{Palabras Clave}: Fractales, Morfología de Galaxias, Galaxia Espiral , Automata Celular.\\
\end{abstract}

\section{Introducción}

\section{Conceptos Básicos}
En esta sección daremos los conocimientos mínimos para poder abarcar este trabajo...
\subsection{Qué es un Automata Celular y cómo es aplicable a galaxias}
\subsubsection{Autómatas Celulares}
Los AC proveen un modelo matemático, simple, discreto y deterministico para sistemas biológicos, físicos computacionales y más.
A pesar de su simple construcción, los AC han demostrado ser capaces de reproducir/generar comportamientos complejos.~\cite{Wolfram1982}
Un Autómata Celular se puede definir como un sistema dinámico donde el espacio y el tiempo son discretos. Describen la evolución de un sistema espacialmente 
explícito en función de un conjunto de reglas de evolución de estado, que determinan el cambio de estado de cada celda que 
constituye el espacio, en función de su propio estado y el estado de un conjunto de celdas vecinas llamadas vecindario.~\cite{Cesimo2004}
\subsection{Recordando que son las coordenas polares y esféricas}
Para desarrollar nuestro AC con vecindad radial, utilizamos coordenadas polares para la identificación de cada \emph{célula} en 2D y
coordenadas esféricas para 3D.\\
En esta sección daremos las definciones y las ecuaciones para el pasaje a coordenadas cartesianas que fueron empleadas para el ploteo.
\subsubsection{Coordenadas polares}
Es un sistema de coordenadas bidimensional en el cual cada punto del plano se determina por un ángulo y una distancia.\\
De manera más precisa, se toman: un punto $O$ del plano, al que se le llama origen o polo, y una recta dirigida (o rayo, o segmento OL) 
que pasa por $O$, llamada eje polar (equivalente al eje x del sistema cartesiano), como sistema de referencia. Con este 
sistema de referencia y una unidad de medida métrica (para poder asignar distancias entre cada par de puntos del plano), todo 
punto $P$ del plano corresponde a un par ordenado $(r, \theta)$ donde $r$ es la distancia de $P$ al origen y $\theta$ es el 
ángulo formado entre el eje polar y la recta dirigida $OP$ que va de $O$ a $P$.\\
Definido un punto en coordenadas polares por su ángulo $\theta$ sobre el eje x, y su distancia $r$ al centro de coordenadas, 
se tiene:
\begin{equation}
x= r \cos \theta \;\;
y= r \sen \theta    
\end{equation}
Esta conversión nos permite plotear sobre el plano $xy$.\\
También nombraremos las coordenadas esféricas ya que luego implementaremos nuestra aplicación en 3D.\\
Las coordenadas esféricas pueden convertirse en coordenadas cartesianas de la siguiente manera:
Las coordenadas polares también pueden extenderse a tres dimensiones usando las coordenadas $(\rho,\; \varphi,\; \theta)$, 
donde $\rho$ es la distancia al origen, $\varphi$ es el ángulo con respecto al eje z (medido de 0\textordmasculine a 180\textordmasculine), 
y $\theta$ es el ángulo con respecto al eje x (igual que en las coordenadas polares, entre 0\textordmasculine y 360\textordmasculine) Este sistema de 
coordenadas es similar al sistema utilizado para denotar la altitud y la latitud de un punto en la superficie de la Tierra, 
donde se sitúa el origen en el centro de la Tierra, la latitud $\delta$ es el ángulo complementario de $\varphi$ (es decir, $\delta = 90 - \varphi)$, y la 
longitud $l$ viene dada por $\theta - 180^\circ .12$
Las coordenadas esféricas pueden convertirse en coordenadas cartesianas de la siguiente manera:
\begin{equation}
x = \rho  \sen\phi  \cos\theta \;\;
y = \rho  \sen\phi  \sen\theta \;\;
z = \rho  \cos\phi
\end{equation}

\begin{center}
\fbox{Todo lo que concierne a definiciones de coordenadas fue sacado de wiiikiii}
\end{center}

\section{AC en Galaxias}
Un autómata celular en 2D puede ser usado como un eficiente constructor de galaxias y una herramienta para observar su evolución. 
~\cite{kent}
Este modelo mantiene la estructura espiral usando como entrada datos sobre formación estelar, posición, tamaño y velocidad y 
la rotación de la galaxia.~\cite{kent}
\subsection{Pasos para la construcción de una galaxia}
\begin{itemize}
    \item Las celdas pueden tener 15 posibles estados (vida de una formación estelar/dt), con cero denotamos la 
    ausencia de estrellas o formaciones.
    \item Dada una celda activa, se examina su vecindad y estas tienen una probabilidad de iniciarse de 16 a 22\% ~\cite{kent}, en este
    trabajo particularmente se trabajo con 18\%.
    \item Cada celda tendrá tres datos cruciales, radio y ángulo, para detectar su posición y un estado que nos indica en que
    etapa se encuentra nuestra formación.
    \item Luego de examinarlas incremetamos en uno a todas las estrellas activas para denotar su edad que se verá reflejada en
    una paleta de colores, y aplicamos una función de rotación.  
\end{itemize}
En la sección anterior hablabamos de coordenadas polares, en este punto podemos vislumbrar cual será su utilidad.
Para poder evaluar cada celda debemos conocer su posición y ya que hablamos de generar galaxias del tipo espiral estas coordenadas
demuestran ser las indicadas.
    \begin{figure}
        \begin{center}
            \includegraphics[scale=0.5]{Imagenes/vecindad.png}
        \caption{Ejemplo de vecindad tomada para la generación de galaxias.}
         \label{fig:vecindad}
    \end{center}
    \end{figure}

En la Figura ~\ref{fig:vecindad} podemos observar un ejemplo de la vecindad a la que nos referimos.
El modelo tomado de ~\cite{Schulman} que consta de N círculos concéntricos, cada anillo es divido en $6N$ regiones.\\
Cada celda tiene normalmente 6 vecinos, si la celda se encuentra activa, es decir, si se generó una formación de estrellas,
a medida que avance la simulación también lo harán sus estado (tiempo de vida o mejor dicho edad).\\[0.2cm]

\subsection{Patrones de galaxias espirales}
La obtención de los patrones no es  del todo trivial, ya que estas no rotan en forma rígida.
La velocidad de rotación varía rápidamente cerca del centro de la galaxia y en radios superiores se vuelve bastante 
plana(?).\\[0.2cm] En toda la zona perteneciente a los brazos de la galaxia la velocidad circular es independiente del radio, lo que nos indica 
que la velocidad circular es independiente del radio y que la velocidad angular no es constante y decrese en forma de $1/r$.
~\cite{Schulman}.
  
\section{Mirando un poco más arriba}
En general, las galaxias no están aisladas en el espacio sino que suelen ser miembros de agrupaciones de tamaño pequeño o medio, que 
a su vez forman grandes cúmulos de galaxias. Nuestra galaxia pertenece a una agrupación pequeña de unas 20 galaxias que los astrónomos 
llaman el Grupo Local. La Vía Láctea y la galaxia Andrómeda son los dos miembros mayores, con 100.000 o 200.000 millones de estrellas 
cada una.\\[0.2cm]
Todos estos cúmulos se mueven en la misma dirección; la razón de esto podría ser otro supercúmulo escondido a la vista por nuestra propia 
galaxia, ya que se tiene conocimiento de supercúmulos a una distancia de hasta 300 millones de años luz. 
Por lo general, la distribución de cúmulos y supercúmulos en el Universo no es uniforme, sino que supercúmulos de decenas de miles 
de galaxias están dispuestos en largos filamentos, fibrosos y con forma de lazo, separados por grandes vacíos.\\[0.2cm]
No existen explicaciones consensuadas para la formación de galaxias o la distribución de las mismas en el universo. El modelo estándar 
que describe el Big-bang y la evolución del Universo es el modelo de Friedman-Robertson-Walker (FRW). Basado en las ecuaciones de la 
relatividad general, el modelo FRW asume que el Universo es homogéneo, todos los lugares del cosmos son idénticos y comparten iguales 
condiciones físicas, esta homogeneidad se ve reflejada en escalas astronómicas inmensas (mayores que los 100 millones de años-luz) ~\cite{Gangui2009}
e isótropo, para un observador $X$ el universo presenta iguales características en todas las direcciones. 
Asunción que ha venido en denominarse Principio Cosmológico.~\cite{fractals}\\[0.2cm]
Recientemente ha surgido una clara imagen de la acumulación jerárquica de galaxias en cúmulos y supercúmulos.
Argumentan que la aglomeración jerárquica de las galaxias continúa incluso para las grandes escalas, que el universo 
exhibe una distribución de materia autosimilar a todas las escalas. Si medimos el número de galaxias vecinas $N(R)$, a menos de un radio $R$ 
alrededor de una galaxia escogida al azar, una distribución fractal nos ofrecería:
$N(r) = cte * RD$
donde D es la dimensión fractal y en un espacio euclídeo de dimensión 3 estaría entre 0 y 3. Si el valor D fuera 3 significaría que las 
galaxias se distribuyen homogéneamente, como supone el punto de vista convencional. Las estimaciones indican sin embargo que esta 
dimensión es aproximadamente $D = 2.1$. De modo que la distribución de galaxias en el universo es fractal, al menos hastas las escalas que 
nos permiten explorar con detalle nuestros instrumentos (unos 300 millones de años luz). ~\cite{fractals}.

Desde Peebles(1980) y Mandelbrot(1982) se sabe que la distribución de galaxias en escalas pequeñas es fractal lo que muchos investigadores tratan de 
demostrar es que a grandes escalas sucede lo mismo por lo que sería invariante respecto a su escala.

\subsection{Modelo Fractal de Charlier}
Charlier vió al universo como fractal con galaxias como unidades del mismo, cada una contiene $N_1$ estrellas.\\
\todo{Todo esto es de el paper THE CONCEPT OF FRACTAL COSMOS}
Luego se definen un \emph{cluster} de galaxias, $G_1$ una galaxia con $N_2$ componentes, luego un \emph{cluster} de \emph{clusters}
$G_2$ con $N_2$ elementos. Cada $G_i$ cuenta con un radio $R_i$ y una masa correspondiente $M_i$.
En este modelo se asume por simplicidad que todos son cuerpos esféricos por lo que cada unidad $G_i$ esta compuesta de
cierta manera talque el radio $p_i$ esta definido por:
\begin{equation}
    p_i= {{R_i }\over{N^{1/3}}}
\end{equation}
por lo que la distancia promedio entre dos vecinos cercanos en $G_i$ es $2p_i$ y $M_i=N_iM_{i-1}= N_iM_{i-1}N{i-2}...N_2N_1M_0$

Estas definiciones estan dadas de esta forma, ya que eliminan ambas paradojas (de \emph{Seeliger}\footnote{de que se trata} y \emph{Olbers}\footnote{de que se trata}) quedan 
anuladas, la justificacion de estas queda fuera del alcance de este trabajo.
\subsection{Vista fractalosa del universo}

Ahora veamos desde una visión más fractalosa y simple de nuestro universo. Como se ve en /todo{mismo paper}. Comparemos los elementos
de la geometría clásica con fractales.

\section{Implementación: \emph{GalaPy}}
Para este trabajo se realizo una escueta simulación desde el punto de vista morfológico de las galaxias. Por cuestiones de conocimiento
previo se utilizó para el desarrollo el lenguaje de programación \emph{Python} y como biblioteca de ploteo \emph{Mayavi}. A continuación 
detallaremos todas las herramientas/bibliotecas utilizadas.
\begin{description}
    \item[Python] Python es un lenguaje de programación que permite trabajar con mayor rapidez e integrar sistemas con mayor eficacia. 
    Se puede aprender a usar Python y obtener beneficios casi inmediatos en la productividad y reducir los costos de mantenimiento.\\
    Para más información \url{http://python.org/}
    \item[Prymatex] Es un editor basado en PyQt, que soporta Bundles de TextMate, contiene Snippets, resaltado de sintaxis, entre otras
        bondades.\\
        Para más información \url{http://prymatex.org/}  
    \item[Ipython] Es un shell interactivo que añade funcionalidades extra al modo interactivo incluido con Python, tales como resaltado de líneas y errores mediante colores, una sintaxis adicional para el shell, autocompletado mediante tabulador de variables, módulos y atributos; entre otras. Es un componente del paquete SciPy.\\
    Para más información  \url{http://ipython.org/}
    \item[NumPy] Permite manipular de manera rápida y eficiente arreglos numéricos tales como vectores y matrices (pero también arreglos multidimensionales de rango arbitrario).\\
       Para más información \url{http://numpy.scipy.org/}
    \item[SciPy] SciPy contiene módulos para optimización, álgebra lineal, integración, interpolación, 
       funciones especiales, FFT, procesamiento de señales y de imagen, resolución de ODEs y otras tareas 
       para la ciencia e ingeniería. Está dirigida al mismo tipo de usuarios que los de aplicaciones como MATLAB, GNU Octave, y Scilab.\\
       Para más información \url{http://www.scipy.org/}
    \item[Mayavi] Mayavi2 es un programa interactivo (en forma opcional) que permite elaborar gráficos en 3D de los datos 
    científicos en Python con scipy. Es el sucesor de MayaVi para la visualización 3D.\\
    Para más información \url{http://mayavi.sourceforge.net/}
\end{description}
\subsection{Tips pa' remarcar}
Para la distribución de las  estrellas tomamos la función detallada en ~\cite{kent}. En el código se puede ver flejada en
\emph{distributionFuncStars}:
\begin{small}
\begin{minted}{python}
    ....
    ....
    
    class Galaxy:
    """Clase galaxia"""

    def __init__(self, radio, neighborhood):
        """Initialize the galaxy with some radius
        and the chosen neighborhood """

        self.distributionFuncStars = dis.rv_discrete(name='custm', values=[(0, 1),
                                                (0.84, 0.16)])
        self.distributionGas = dis.rv_discrete(name='custm_gas', values=[(0,1),(0.3,0.7)])
    ....
    ....
\end{minted}
\end{small}
Para agregar un poco más de aleatoriedad se agrego la funcion de distribución \emph{distributionGas}. que em um comienzo se 
trató de tomar la función "real" pero se tuvo que modificar para que esta aleatoriedad aparezca. Podemos ver los resultados
generados tomando estas funciones dedistribución en la figura ~\ref{fig:gala_1}
\begin{figure}
        \begin{center}
            \includegraphics[scale=0.3]{outputs/galaxia_1.png}
        \caption{Galaxia generada con \emph{GalaPy}.}
         \label{fig:gala_1}
        \end{center}
\end{figure}
Para reflejar la velocidad angular explicada en la sección anterior se implemento dentro de la clase \emph{galaxia} la siguiente función.
\begin{small}
    \begin{minted}{python}
        .... ... 
        
     def growthFunction(self, star):
        """Increases growth of each star"""
        if star.state > 15:
            star.state = 0
        star.state += 1
        try:
            #set angular velocity
            star.angle = star.angle + 1 / float(star.r)    
        ....
        ....
        ....
    \end{minted}
\end{small}
La forma en que se rastrillo la galaxia, es decir, ver el estado de los vecinos aledaños a un cúmulo estelar, para saber si se 
debia crear un nuevo cúmulo o no, se muestra a continuación en el código:
\begin{small}
    \begin{minted}{python}
        ....
        ....
        
     def scanning(self):
        """See state of stars and change formations"""
        for s in self.stars:
            localNeighborhood = ifilter(lambda star: star[1] == star[0].r + 1,
                                        izip(self.stars, repeat(s.r, len(self.stars))))
            map(self.birthFunction, localNeighborhood)
        #aqui envejecemos la galaxia
        map(self.growthFunction, self.stars)
        ....
    \end{minted}
\end{small}
\section{Conclusiones}

\begin{small}
\begin{thebibliography}{1}
\bibitem[Cesimo 2004]{Cesimo2004} Cesimo - Biodesus, Facultad de Ingeniería - Facultad de Ciencias Forestales y Ambientales. Universidad de Los Andes, 2004.
\bibitem[Wolfram 1982]{Wolfram1982} Cellular Automata as Simple Self-Organizing Systems, 1982.
\bibitem[Schulman xxx]{Schulman} P. E. Seiden and L. S. Schulman.
\bibitem[Kent xxx]{kent}Dr. Brian R. Kent, National	Radio Astronomy Observatory, Morphology and Dynamics of Galaxies with Cellular Automata.
	
  

\bibitem[dmae]{fractals}\url{http://www.dmae.upm.es/cursofractales/capitulo8/frames.htm}
\end{thebibliography}
\end{small}


\end{document}

